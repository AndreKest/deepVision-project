\chapter{Vorlagen}

% Literaturverzeichnis \cite{label}
% Abbildungen, Tabellen, Listing, ... \autoref{label}


\section{Abbildungen}
% small h -> Latex legt fest wo Abbildung am besten liegt wegen Seitenaufteilung
% big H -> Abbildung genau an dieser stelle
\begin{figure}[h]
	\centering
	\includegraphics[width=0.55\linewidth]{sinus_plot.pdf}
	\caption[Beispiel Plot der Matplotlib Bibliothek]{Beispiel Plot der Matplotlib Bibliothek. Quelle: Eigene Aufnahme}
	\label{fig:sinus_plot}
\end{figure}


\section{Tabelle}
% l: linksbündig, c: zentriert, r: rechtsbündig
\begin{table}[H]
	\centering
	\begin{tabular}{l|c|c|c|c|c|c}
		&	Saplte 1	&	Spalte 2	&	Spalte 3	&	Spalte 4	&	Spalte 5	&	Spalte 6\\
		\hline
		Zeile 1		&	1	&	1	&	1 	&	1 &	1	&	1\\
		\hline
		Zeile 2	&	1	&	1	&	1	&	1	&	1		&	1\\
		\hline
		Zeile 3	&  1	&	1	&	1	&	1	&	1	&	1\\
		\hline
		Zeile 4	&	1	&	1	&	1	&	1	&	1	&	1
	\end{tabular}
\end{table}

\section{Programmiercode}
\lstinputlisting[caption=Matplotlib Beispiel Code. Quelle: Eigener Programmcode, label=code_Matplotlib, captionpos=b, firstline=1, lastline=11]{code/Matplotlib_Example.py}

\section{Mathematik}
\subsection{Matrix}
\begin{align}
	M =
	\begin{pmatrix}
		1 &	1	& 1 & 1\\
		1 & 1 & 1 & 1\\
		1 & 1 & 1 & 1\\
		1 & 1 & 1 & 1
	\end{pmatrix}
\end{align}


\subsection{Formel}

\begin{align}
	\begin{split}
		R(\alpha, \beta, \gamma) &=  R_z(\alpha) \cdot R_y(\beta) \cdot R_x(\gamma)\\
		&=\begin{pmatrix}
			C_\alpha\cdot C_\beta & -S_\alpha \cdot C_\gamma + C_\alpha \cdot S_\beta \cdot S_\gamma  & S_\alpha \cdot S_\gamma + C_\alpha \cdot S_\beta \cdot C_\gamma  \\
			S_\alpha\cdot C_\beta & C_\alpha\cdot C_\gamma + S_\alpha\cdot S_\beta\cdot S_\gamma & -C_\alpha\cdot S_\gamma + S_\alpha \cdot S_\beta \cdot C_\gamma \\
			-S_\beta & C_\beta \cdot S_\gamma & C_\beta \cdot C_\gamma
		\end{pmatrix}.
	\end{split}
	\label{EulerToRot}
\end{align}

\begin{equation*}
	[\bar{u}]_x \coloneqq
	\begin{pmatrix}
		0 & -\bar{u}_3 & \bar{u}_2 \\
		\bar{u}_3 & 0 & -\bar{u}_1 \\
		-\bar{u}_2 & \bar{u}_1 & 0 \\
	\end{pmatrix}
\end{equation*}

\begin{align}
	\label{VecToRot1}
	R = I + \sin(\theta)\cdot [\bar{u}]_x + (1-\cos(\theta))\cdot [\bar{u}]_x^2
\end{align}

\section{Aufzählung}
\begin{itemize}
	\item Item 1
	\item Item 2
	\item Item 3
	\item Item 4
\end{itemize}




\section{Station X: Zaun}
\subsection{Animation}
%Skirpt, Animator, Zustände zugreifen

\section{Weiteres}
%Name überlegen für Kapitel
%Animation für Igel Mutter -> von Magdalena -> Blend Tree 


