\chapter{Einleitung}

\section{Aufgabenstellung}
Im Rahmen des Studiengangs Ausgewählte Themen AR/VR soll ein Projekt im Themenbereich Augmented Reality (AR), Virtual Reality (VR) oder Mixed Reality (MR) bearbeitet werden. Die Bearbeitung des eigens ausgesuchten Projekts soll in einer Gruppe erfolgen. Die Gruppe für dieses Projekt besteht aus folgenden Personen: Johannes Horst, Saniye Ogul, Stefanie Hofmann, Magdalena Bienefeld und mir (André Kestler). Als Projekt wurde ein Igel Simulator als VR-Spiel entworfen und programmiert. Mit der VR-Brille Oculus Quest kann das Spiel an einem Computer mithilfe des Link-Kabels gespielt werden. 


\section{Projektmanagement}
\subsection{GitHub}
Zur Versionsverwaltung des Projekts wurde GitHub verwendet. Dort hat Johannes Horst ein Repository für das Projekt angelegt und jeder Teilnehmer hat seine Arbeiten dort einchecken können. Außerdem ist durch dieses Tool jeder auf dem aktuellsten Softwarestand und Fehler können einfach überblickt werden. \cite{GitHub}


\subsection{Trello}
Bei Trello handelt es sich um einen Onlinedienst, der ein Projekt bei der Verwaltung und Verteilung von Aufgaben unterstützt. Durch ihn lassen sich Arbeitsboards anlegen und in diesem Listen für bestimmte Themengebiete. In den Listen können Karten für die zu bearbeitenden Aufgaben erstellt werden und Benutzer zugeordnet werden. Der Vorteil ist, dass man immer einen Überblick über die Aufgaben hat und zum Ende des Projekts eine Übersicht über die bearbeiteten Themen hat. \cite{Trello}

\subsection{Discord}
Zur Absprache und Klärung von Problemen wurde auf das Kommunikationstool Discord zurückgegriffen. Darin wurden wöchentlich Meetings angesetzt, um den Zwischenstand und die Aufgabenverteilung festzulegen.

\subsection{Unity}
Für das zu bearbeitende Thema hat sich meine Gruppe für, die in der Vorlesung kennengelernte Spielentwicklungsumgebung, Unity entschieden. Die Anwendung wurde mit der Unity Version 2020.3.20f1 entwickelt.  \cite{Unity}

