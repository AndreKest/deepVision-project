\chapter{Einleitung}

\section{Aufgabenstellung}
Im Rahmen der Vorlesung Deep Vision ist ein Projekt im Themenbereich des Kurses zu bearbeiten. Die Bearbeitung erfolgt als Einzelarbeit. Als Projekt werden zwei YOLO (You only look once) Netzwerke mit dem Udacity Self Driving Car Datensatz trainiert und miteinander verglichen. In der folgenden Arbeit werden YOLOX und YOLOv8 verwendet.


\section{Übersicht}
In Kapitel \ref{chap:dataset} wird zunächst der Datensatz beschrieben. Dabei wird auf die Klasseneinteilung und die Datenstruktur eingegangen. In Kapitel \ref{chap:yolox} wird das YOLOX-Netzwerk vorgestellt. Dabei wird auf die verwendete Verlustfunktion, die Architektur und die Methoden eingegangen. Kapitel \ref{chap:yolov8} beschreibt die verwendete Architektur von YOLOv8. In Kapitel \ref{chap:metrics} werden die Ergebnisse mit dem Datensatzes vorgestellt. Das letzte Kapitel befasst sich mit einer Zusammenfassung über die Arbeit und gibt einen Ausblick über die weitere Bearbeitung. Im \nameref{chap:appendix} wird beschrieben, wie die angegebenen Skripte verwendet werden, um die Netzwerke selbst zu trainieren.


