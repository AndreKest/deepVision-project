\chapter{Zusammenfassung und Ausblick}\label{chap:conclusion}
In der vorliegenden Arbeit wurde ein Vergleich zwischen den Modellen YOLOX und YOLOv8 anhand des Udacity Self Driving Car Datensatzes durchgeführt. Beide Modelle weisen eine ähnliche Architektur auf und benutzen ähnliche Methoden zur Verarbeitung der Daten (gleiche Datenvorverarbeitung, Labelzuweisung, Decoupled Head, Anchor-Free-Detection). Der Vorteil von YOLO8 besteht darin, dass es einfach anzuwenden ist, indem es mit pip das  \textit{ultralytics}-Paket installiert.

Für zukünftige Arbeiten wird empfohlen, den Datensatz anzupassen, um eine ausgewogenere Verteilung der Klassen zu erreichen. Derzeit gibt es eine übermäßige Anzahl von Objekten der Klasse \textit{car}. Durch eine Anpassung des Datensatzes kann die Leistung der Modelle weiter verbessert werden, insbesondere wenn es darum geht, andere Objektklassen korrekt zu erkennen.

Insgesamt bieten sowohl YOLOX als auch YOLOv8 vielversprechende Ansätze für die Objekterkennung im Bereich des autonomen Fahrens. Durch weitere Optimierungen und Anpassungen des Datensatzes können die Ergebnisse noch weiter verbessert werden, um eine zuverlässige und genaue Erkennung von Verkehrssituationen zu ermöglichen.