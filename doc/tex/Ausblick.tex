\chapter{Zusammenfassung und Ausblick}
\section{Zusammenfassung}
Die Entwicklung eines Spiels ist ein komplexer Prozess. Der Entwurf bezieht nicht nur die Implementierung der Logik mit ein, sondern im Laufe der Entwicklung müssen noch weitere Aufgaben bearbeitet werden. Somit musste für das Implementieren der Logik die Programmiersprache C\# erlernt werden. Für das Erstellen der Objekte habe ich einen Einblick in die 3D-Modellierung mit Blender bekommen. Besonders der Entwicklungsprozess der Gegenstände beansprucht abhängig vom Detailgrad der Modellierung ein hohes Maß an Zeit. Durch das Verwenden von Soundeffekten musste man sich zudem noch mit Lizenzen auseinandersetzen.


\section{Ausblick}
Das Spiel lässt sich in vielerlei Hinsicht weiterentwickeln. Zum einen können weiter Stationen implementiert werden, welche das Leben eines Igels noch weiter vertiefen oder die bereits bestehenden Stationen noch verbessert werden. Somit kann die Station des Rasenmähers auf die gesamte Spielwelt ausgeweitet werden und der Roboter mit einem Pathfinding Algorithmus ausgestattet werden. Dieser Algorithmus ermöglicht es den Roboter, unabhängig von der Position des Spielers, jagt auf diesen zu machen. Der Anwender müsste dann auf den Roboter reagieren und sich vor dem Objekt verstecken. Die Station der Straße könnte durch Collider erweitert werden, um einen Aufprall mit dem Igel zu registrieren. Zum erfolgreichen Abschließen der Station müsste es der Spieler auf die gegenüberliegende Seite der Straße schaffen. Daraus könnte sich eine Art Minigame entwickeln, dass den Autos ausgewichen werden muss. Außerdem könnte das Spiel um weitere Level mit anderen Umgebungen erweitert werden. In dem Umfang, dass man sich zum einen in der Stadt zurechtfinden muss und zum anderen auf dem Land.



