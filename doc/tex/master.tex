% LaTeX-Vorlage zur Erstellung einer Abschlussarbeit in der Fakultät Elektrotechnik, Medien und Informatik an der OTH Amberg-Weiden
% Diese Vorlage entstand im Rahmen des Kurses "LaTeX fürs Studium"
% Aktuelle Version: v0.03bacsem
% Stand: 18.04.2020
%
% Changelog:
%
% v0.03bacsem-us: 28.04.2020, Grafikpfad und Bibliographiemanagement 
%                             via bibtex hinzugefügt
% Grafikpfad für das graphicx-Paket:
%   \graphicspath{{images/}} % hinter \usepackage{graphicx}
% Literaturverzeichnis nach DIN:          
%   \usepackage[square,numbers,sort]{natbib} 
% Literaturverzeichnis am Ende:
%   \bibliographystyle{natdin}
%   \bibliography{literatur}
% 
% v0.02: 06.08.2015, Anpassung der Vorlage:
% + Persönliche Informationen (Vorname, Name, Titel usw.) werden direkt in die PDF-Dokumenteinstellungen übernommen
% + Korrektur der Verlinkung von Abbildungs- und Tabellenverzeichnis aus dem Inhaltsverzeichnis (phantomsection) bzw. deren Seitenzahl
%   Besten Dank für diesen Hinweis an Jan-Olaf Becker
% + Anpassung des Namens der Fakultät nach deren Umbenennung
%
% v0.01: 14.03.2012, Erstellung der Vorlage
% oneside: legt Seitennummerierung in die Mitte -> einseitiger Druck
% twoside: legt Seitennummerierung abh. von Seite an rechten/linken Rand -> vorder-/rueckseite Druck
\documentclass[12pt,oneside]{report}
\usepackage[T1]{fontenc}		% Einstellungen fuer Umlaute usw.
\usepackage[utf8x]{inputenc}
\usepackage[ngerman]{babel}

\usepackage{parskip}			% Einstellungen fuer Absaetze: Abstand statt Einrueckung

\usepackage[a4paper,			% Papierformat A4
	    left=2.0cm,				% linker Rand
	    right=2.0cm,			% rechter Rand
	    top=2.0cm,				% oberer Rand
	    bottom=2.0cm,			% unter Rand
	    marginparsep=5mm,		% Abstand der Randnotizen
	    marginparwidth=10mm, 	% Breite der Randnotizen
	    headheight=7mm,			% Hoehe der Kopfzeile
	    headsep=1.2cm,			% Abstand der Kopfzeile
	    footskip=1.5cm,			% Abstand der Fusszeile
	    includeheadfoot]{geometry}

\usepackage{fancyhdr}						% Konfiguration von Kopf- und Fusszeilen
\pagestyle{fancy}							% Seitenstil 'fancy'
\fancyhf{}									% vorhandene Einstellungen loeschen
\setlength{\headwidth}{\textwidth}			% Kopf- und Fusszeile so breit wie der Haupttext
\fancyfoot[R]{\thepage} 					% Festlegung des Seitenstils: Seitenzahlen in der Fusszeile rechts
\fancyfoot[L]{\leftmark}					% Kapitelnr. und -Bezeichnung in der Fusszeile links
\fancyhead[R]{\IhreArbeit}					% "Bachelorarbeit" in der Kopfzeile rechts
\fancyhead[L]{\IhrVorname\ \IhrNachname}	% Vorname und Name in der Kopfzeile links
\renewcommand{\chaptermark}[1]{			% Definition der Ausgabe des Kapitels
  \markboth{Kapitel \thechapter. #1}{}}
\renewcommand{\headrulewidth}{0.5pt}		% Trennlinie zwischen Kopfzeile und Haupttext
\renewcommand{\footrulewidth}{0.5pt}		% Trennlinie zwischen Haupttext und Fusszeile
\fancypagestyle{plain}{					% Anpassung des Seitenstils 'plain' bei Beginn neuer Kapitel
  \fancyhf{}								% Vorbelegung loeschen
  \fancyfoot[C]{\thepage}					% Seitenzeilen in der Fusszeile mittig
  \fancyhead[R]{\IhreArbeit}				% "Bachelorarbeit" in der Kopfzeile rechts
  \fancyhead[L]{\IhrVorname\ \IhrNachname}	% Vorname und Name in der Kopfzeile links
}

%\usepackage{amsmath}			% Pakete fuer den Mathematikmodus
%\usepackage{amssymb}
\usepackage[intlimits]{empheq}

\usepackage[sc]{mathpazo}		% Schriftart Palatino fuer Haupttext und Mathematikmodus
\usepackage{pifont}				% zusaetzliche Symbole

\usepackage{setspace}
\setstretch{1.25}

\usepackage[format=hang,		% Einstellung fuer Bildunterschriften
            font={footnotesize},
            labelfont={bf},
            margin=1cm,
            aboveskip=5pt,
            position=bottom]{caption}

\usepackage{graphicx}	   % Einbinden von Grafiken (jpg, png, pdf, ...)
\graphicspath{{images/}}   % Suchpfad für Grafikdateien

\usepackage[svgnames,table,hyperref]{xcolor} 	% Verwendung von Farben
\usepackage{tikz}								% Erstellen von Grafiken
\usetikzlibrary{positioning,arrows,plotmarks} % TikZ-Bibliotheken
%\usepackage{pgfplots}                           % Darstellung von Plots, Funktionen, Graphen usw.

%
% Weitere Pakete
%
\usepackage{listings}			% Darstellung von Quellcode
\definecolor{mygreen}{rgb}{0,0.6,0}
\definecolor{myblue}{rgb}{0.0,0.0,1.0}
\definecolor{myred}{rgb}{1.0,0.3,0.0}

\DeclareMathOperator{\arctantwo}{arctan2}

\usepackage{leftidx}

\lstset{language = Python,
	numbers = none,
	basicstyle = \small\ttfamily,
	keywordstyle = \color{myblue},
	commentstyle = \color{mygreen},
	stringstyle = \color{myred},
	columns = flexible,
	showstringspaces = false,
	frame = single,
	morekeywords={as, family}
}
\usepackage{float}
\usepackage[square,numbers,sort]{natbib} % Referenzen
%
%\usepackage[european, siunitx]{circuitikz}	% Darstellung von Schaltungen
%
%\usepackage{enumerate}			% Formatierung nummerierter Listen

\usepackage{microtype,relsize}					% Wird verwendet, um Nachnamen auf Titelseite gesperrt darzustellen
\newcommand*{\Sperren}[1]{\textls*[100]{#1}}

% 
% Persoenliche Angaben
% 
\newcommand*{\IhrVorname}{André}
\newcommand*{\IhrNachname}{Kestler}
\newcommand*{\IhrStudiengang}{Künstliche Intelligenz}
% Bachelorarbeit, Masterarbeit, Studienarbeit
\newcommand*{\IhreArbeit}{Studienarbeit AR/VR}
\newcommand*{\IhrTitelDE}{Der Igel Simulator}
% \newcommand*{\IhrTitelEN}{Title of the study work}
\newcommand*{\IhrBearbeitungszeitraumVON}{01. Dezember 2021}
\newcommand*{\IhrBearbeitungszeitraumBIS}{02. Februar 2022}
\newcommand*{\IhrErstpruefer}{Prof. Dr.-Ing. Gerald Pirkl}
% \newcommand*{\IhrZweitpruefer}{Prof. Dr.-Ing. Franz Klug}
% \newcommand*{\IhreFirma}{OTH Amberg-Weiden}
% \newcommand*{\IhrFirmenbetreuer}{Prof. Dr.-Ing. Matthias Wenk}
\newcommand*{\IhreZusammenfassung}{Zusammenfassung}

\newcommand*{\IhreSchluesselwoerter}{Schlüsselwortliste}

\newcommand*{\IhrAbstract}{Abstract}



\usepackage[bookmarks, raiselinks, pageanchor, % PDF-Einstellungen
            hyperindex, colorlinks,
            citecolor=black, linkcolor=black,
            urlcolor=black, filecolor=black,
            menucolor=black]{hyperref}
\hypersetup{pdftitle={\IhrTitelDE},%
            pdfauthor={\IhrVorname\ \IhrNachname},%
            pdfsubject={\IhreArbeit},%
            pdfkeywords={\IhreSchluesselwoerter}}

%
% Beginn des Textteils
%
\begin{document}
    \pagenumbering{roman}
  \thispagestyle{empty}
  \begin{center}
    \Large
    Ostbayerische Technische Hochschule Amberg-Weiden\\
    Fakultät Elektrotechnik, Medien und Informatik\\[1cm]
    Studiengang \IhrStudiengang\\[1cm]
    \textbf{\IhreArbeit}\\[1cm]
    von\\[1cm]
    \IhrVorname\ \Sperren{\textbf{\IhrNachname}}\\[1cm]
    \textbf{\IhrTitelDE}\\[1cm]
%    \IhrTitelE
  \end{center}
  \vspace*{2.5cm}
  \begin{tabbing}
    \underbar{Bearbeitungszeitraum:}\qquad\= von\qquad\=\IhrBearbeitungszeitraumVON\\
                                          \> bis      \>\IhrBearbeitungszeitraumBIS
  \end{tabbing}
  \vspace*{1cm}
  \underbar{1. Prüfer:}\qquad\IhrErstpruefer\par 
 % \underbar{2. Prüfer:}\qquad\IhrZweitpruefer
    \tableofcontents
    \thispagestyle{empty}
 	\newpage
  
% ----------------------------------------------------------------------  
  
%  \chapter*{Symbole, Formelzeichen und Einheiten}
 % \begin{tabular}{ll}
 % 	%$I$ & Einheitsmatrix\\
 % 	%GB     & Gigabyte ($10^9$ Bytes)\\
 % \end{tabular}
 % \newpage
  
% ----------------------------------------------------------------------  
  \chapter*{Abkürzungsverzeichnis}
    \begin{tabular}{ll}
    	%KI	& Künstliche Intelligenz\\
    	%TCP & Tool Center Point\\  		
    	AR & Augmented Reality\\
    	VR & Virtual Reality\\
    	MR & Mixed Reality
  	\end{tabular}

  \newpage
  \pagenumbering{arabic}
  % Kapitel einfügen
  \chapter{Einleitung}

\section{Aufgabenstellung}
Im Rahmen des Studiengangs Ausgewählte Themen AR/VR soll ein Projekt im Themenbereich Augmented Reality (AR), Virtual Reality (VR) oder Mixed Reality (MR) bearbeitet werden. Die Bearbeitung des eigens ausgesuchten Projekts soll in einer Gruppe erfolgen. Die Gruppe für dieses Projekt besteht aus folgenden Personen: Johannes Horst, Saniye Ogul, Stefanie Hofmann, Magdalena Bienefeld und mir (André Kestler). Als Projekt wurde ein Igel Simulator als VR-Spiel entworfen und programmiert. Mit der VR-Brille Oculus Quest kann das Spiel an einem Computer mithilfe des Link-Kabels gespielt werden. 


\section{Projektmanagement}
\subsection{GitHub}
Zur Versionsverwaltung des Projekts wurde GitHub verwendet. Dort hat Johannes Horst ein Repository für das Projekt angelegt und jeder Teilnehmer hat seine Arbeiten dort einchecken können. Außerdem ist durch dieses Tool jeder auf dem aktuellsten Softwarestand und Fehler können einfach überblickt werden. \cite{GitHub}


\subsection{Trello}
Bei Trello handelt es sich um einen Onlinedienst, der ein Projekt bei der Verwaltung und Verteilung von Aufgaben unterstützt. Durch ihn lassen sich Arbeitsboards anlegen und in diesem Listen für bestimmte Themengebiete. In den Listen können Karten für die zu bearbeitenden Aufgaben erstellt werden und Benutzer zugeordnet werden. Der Vorteil ist, dass man immer einen Überblick über die Aufgaben hat und zum Ende des Projekts eine Übersicht über die bearbeiteten Themen hat. \cite{Trello}

\subsection{Discord}
Zur Absprache und Klärung von Problemen wurde auf das Kommunikationstool Discord zurückgegriffen. Darin wurden wöchentlich Meetings angesetzt, um den Zwischenstand und die Aufgabenverteilung festzulegen.

\subsection{Unity}
Für das zu bearbeitende Thema hat sich meine Gruppe für, die in der Vorlesung kennengelernte Spielentwicklungsumgebung, Unity entschieden. Die Anwendung wurde mit der Unity Version 2020.3.20f1 entwickelt.  \cite{Unity}


  % Hauptteil
  \chapter{Spielvorstellung}
\section{Spielmechanik}
%Was soll passieren
%Was ist Sinn hinter dem Spiel
%Karte zu beginn -> Karte des Spiels am Ende -> Evolution der Entwicklung
Die Idee meiner Gruppe, war es ein Spiel zu entwerfen, welches zusätzlich zu den unterhaltenden Faktor auch noch einen lehrenden Einfluss besitzt. In einer virtuellen Welt spielt der Spieler einen Igel in First-Person. Dazu kann sich der Spieler mit den Controllern auf der Karte, welche in \autoref{fig:map} dargestellt ist, fortbewegen. Im Laufe des Spiels lernt der Anwender wissenswertes über Igel und muss sein Wissen bei den verschiedenen Stationen auf die Probe stellen. Das erfolgreiche Bearbeiten der Stationen ist nötig, da man als Errungenschaft jeweils ein Blatt erhält. Wenn alle Blätter gesammelt wurden, wird das Spiel im Haus der Igelfamilie beendet.

\begin{figure}[H]
	\centering
	\includegraphics[width=0.675\linewidth]{map.png}
	\caption[Übersicht über die Spielkarte]{Übersicht über die Spielkarte 1: Tutorial, 2: Essen, 3: Trinken, 4: Zaun, 5: Teich, 6: Feind, 7: Rasenmäher, 8: Straße. Quelle: Eigene Aufnahme}
	\label{fig:map}
\end{figure}
 
Zur Steuerung wurde eine Kombination aus Teleportation und Fortbewegung mit den Joysticks der Controller gewählt. Es wurde keine reine physische Fortbewegung implementiert, da die Karte des Spiels sonst auf die jeweilige begehbare reale Umgebung angepasst werden muss. Damit ist das Spiel auch unabhängig der räumlichen Möglichkeiten spielbar.


\section{Immersion}
Ein Spiel wird als immersiv bezeichnet, wenn der Spieler vollständig in die fiktive Welt eintaucht und dabei die Wahrnehmungen der realen Umgebung sinken oder gänzlich verschwinden. Dies wurde in dem Spiel mithilfe der Größe der modellierten Umgebung versucht zu erreichen. Da ein Igel relativ klein ist, wurde die Welt mit seinen Gegenständen dementsprechend größer modelliert. Dieser Sachverhalt ist in \autoref{fig:big_objects} zu sehen. Für ein schöneres Spielgefühl sind zudem noch Audiobestandteile eingefügt, welche für ein besseres Ambiente sorgen. So sind an der Straße die Autos oder bei der Station des Rasenmähers dieser zu hören.

\begin{figure}[H]
	\centering
	\includegraphics[width=0.55\linewidth]{big_objects.png}
	\caption[Darstellung der Größenunterschiede]{Darstellung der Größenunterschiede. Quelle: Eigene Aufnahme}
	\label{fig:big_objects}
\end{figure}

Es sei aber zu erwähnen, dass durch die Wahl der Fortbewegungsmethode und der Low-Poly Grafik der Effekt der Immersion gemindert wird.

  \include{Hauptteil}
  \chapter{Zusammenfassung und Ausblick}
\section{Zusammenfassung}
Die Entwicklung eines Spiels ist ein komplexer Prozess. Der Entwurf bezieht nicht nur die Implementierung der Logik mit ein, sondern im Laufe der Entwicklung müssen noch weitere Aufgaben bearbeitet werden. Somit musste für das Implementieren der Logik die Programmiersprache C\# erlernt werden. Für das Erstellen der Objekte habe ich einen Einblick in die 3D-Modellierung mit Blender bekommen. Besonders der Entwicklungsprozess der Gegenstände beansprucht abhängig vom Detailgrad der Modellierung ein hohes Maß an Zeit. Durch das Verwenden von Soundeffekten musste man sich zudem noch mit Lizenzen auseinandersetzen.


\section{Ausblick}
Das Spiel lässt sich in vielerlei Hinsicht weiterentwickeln. Zum einen können weiter Stationen implementiert werden, welche das Leben eines Igels noch weiter vertiefen oder die bereits bestehenden Stationen noch verbessert werden. Somit kann die Station des Rasenmähers auf die gesamte Spielwelt ausgeweitet werden und der Roboter mit einem Pathfinding Algorithmus ausgestattet werden. Dieser Algorithmus ermöglicht es den Roboter, unabhängig von der Position des Spielers, jagt auf diesen zu machen. Der Anwender müsste dann auf den Roboter reagieren und sich vor dem Objekt verstecken. Die Station der Straße könnte durch Collider erweitert werden, um einen Aufprall mit dem Igel zu registrieren. Zum erfolgreichen Abschließen der Station müsste es der Spieler auf die gegenüberliegende Seite der Straße schaffen. Daraus könnte sich eine Art Minigame entwickeln, dass den Autos ausgewichen werden muss. Außerdem könnte das Spiel um weitere Level mit anderen Umgebungen erweitert werden. In dem Umfang, dass man sich zum einen in der Stadt zurechtfinden muss und zum anderen auf dem Land.




  %\include{Vorlagen}
  
 
  % Literaturverzeichnis
  \phantomsection
  \addcontentsline{toc}{chapter}{Literaturverzeichnis}
  \bibliographystyle{natdin}
  \bibliography{literatur}
  \newpage
  
  % Abbildungsverzeichnis
  \phantomsection
  \addcontentsline{toc}{chapter}{Abbildungsverzeichnis}
  \listoffigures
  \newpage

  %Tabellenverzeichnis
  %\phantomsection
  %\addcontentsline{toc}{chapter}{Tabellenverzeichnis}
  %\listoftables
  %\newpage
  
  %Anhang
 %\include{anhang}
\end{document}    